The original paper \cite{li2018deep} introduced the notion of prototypes to improve neural networks by providing more explainability. Indeed, we have seen that these prototypes converge to representations that explain why a neural network makes certain decisions. In this paper we ran the model proposed by the original paper and used the exact same hyperparameters as they did. By doing so, we were able to reproduce the results. The reported accuracies between our implementation and their implementation are nearly identical, and the prototypes converge to meaningful and interpretable representations, namely prototypical digits. This indicates that the original model is reproducible. 

Subsequently, we extended the architecture of the original paper by introducting the hierarchical notion of subprototypes. This extension has a number of advantages. First, adding a notion of hierarchy to the network allows for more explainability because we can also discover interclass and intraclass variability of the uncovered classes. This may help researchers further explain why the neural network makes certain predictions.

Second, having multiple prototype layers subdivided in super- and subprototype layers gives our network the flexibility to determine by itself, given the fixed number of total super- and subprototypes, how many subclasses should be discerned from the dataset. This flexibility is constrained in the original paper in which the learned prototypes are forced to be both more general and more specific at the same time \cite{li2018deep}. 

Third, a potential problem with the original prototype network is that it is not guaranteed to learn a representation for each class, as shown in Figure~\ref{fig:non3} in the Appendix. There, the model only learns a single hybrid representation of a 3 and an 8. Our main goal, however, is to provide clear prototypes that allow for better interpretation of why the model makes certain predictions. By fixing the weights for the superprototypes to the negative identity matrix we can ensure that we at least find a representation for each class. In addition to that, having learnable weights for the linear layer connecting the subprototype layer to its corresponding softmax layer, we can still learn the intraclass variability in connection to our superprototypes. With this we can effectively counteract the problem of not finding representations for each class, reducing potential adverse effects on the predictive and interpretative abilities of the model.







However, we also uncovered a number of disadvantages. Firstly, our proposed architecture comes with a number of extra hyperparameters related to the separate loss terms in our loss function that need to be tuned. Generally, there are no efficient ways of performing this optimization \cite{paramoptimization}, so adding more hyperparameters comes at the cost of significantly increased computation time. This added computational cost can be alleviated by using random search instead of full grid search \cite{paramoptimization}.

Secondly, the number of prototypes and in our case subprototypes remains a hyperparameter to be tuned. The addition of subprototypes introduces a plethora of new combinations of the numbers of prototypes per class to be tested and tuned in order to find the right number of (sub)prototypes. Expediting the search for the best number of (sub)prototypes remains an open research question.

Thirdly, the prototypes add explainability, but do not allow for any further post-hoc analysis. This entails that we can not simply use this model on previously created models currently used for decision making to explain its predictions. However, if models such as ours that add explainability as an integral part of their architecture become more prevalent, networks used for predictions may in the future always include a component that provides this explainability, counteracting the need for post-hoc analysis.

Fourthly, the prototypes do not give a feature by feature basis on which one could base an analysis. The prototypes themselves show what is a typical class but do not necessarily show what specific features it looks for in order to do the actual classification. For visual data people can see this relatively easily for themselves but for other data this may not be as apparent. 